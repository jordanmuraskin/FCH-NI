\section*{Discussion}
In this paper we have  shown that linking simultaneously acquired EEG and fMRI using a novel encoding model enables us to probe the high-resolution spatiotemporal dynamics in the human brain. We demonstrate the methodology using neural data acquired during a rapid perceptual decision-making task. Our method, which resolves whole-brain activity with EEG-like temporal resolution, uncovers what appears to be a reactivation of neural activity in perceptual decision making, dynamics  that would otherwise be masked by the temporal averaging and slow dynamics of traditional fMRI. More broadly, our results demonstrate a general non-invasive data-driven methodology for measuring, at high spatiotemporal resolution, latent neural processes underlying human behavior.  Below we discuss the methodology and our specific findings for perceptual decision-making with respect to the current literature.

The trial-to-trial variability we leverage in our methodology can be viewed as a combination of exogenous and endogenous neural variability expressed in the EEG and fMRI BOLD. The exogenous variability is related to the stimulus presentation while the endogenous variability is subject and trial specific. Endogenous trial-to-trial variability during perceptual decisions has been shown to be associated with attention \cite{Walz2013}, reward \cite{Fouragnan2015}, confidence \cite{Gherman2015} and other internal mental states that vary at different times during the perceptual decision. The classifier applied to the EEG data is thus sensitive to both the exogenous variability (stimulus differences) and the endogenous variability at the multiple time steps throughout the decision process. 

The approach we present requires that EEG and BOLD data be collected simultaneously and not in separate sessions in order to exploit the correlations in exogenous and endogenous trial-to-trial variability between EEG and BOLD to temporally ``tag" voxels in the space of the fMRI. To show the importance of collecting the data simultaneously, we ran a control analysis that randomly permuted the trials within their stimulus evidence class, thus effectively simulating an EEG and BOLD dataset collected separately. By destroying the link between the EEG and BOLD trials, the encoding model failed to find any consistent activation (Fig. S8), indicating the necessity of simultaneous acquisition. 

Alternative techniques for fusing simultaneous EEG-fMRI typically do not exploit EEG across the trial and instead only analyze specific ERP components or time windows of interest \cite{DeMartino2010,Fouragnan2015,Goldman2009,Huster2012,Jann2009,Jaspers-Fayer2012,Mayhew2013,Novitskiy2011,Omata2013,Walz2013,Walz2013a,Warbrick2013a,Warbrick2009}. Results from these techniques identify regions that modulate with the specific components, but yield limited information about the timing of other task-relevant regions seen in traditional fMRI contrasts. The methodology developed here extends the previously reported work of Goldman et al. \cite{Goldman2009} and Walz et al. \cite{Walz2013} by combining their EEG data reduction techniques with techniques developed for encoding stimulus features onto BOLD data \cite{Cukur2013,Hansen2007,Kay2008,Naselaris2011,Nishimoto2011,Stansbury2013}, ultimately providing a framework for labeling voxels in task-relevant fMRI contrasts with their timing information (Fig. S2C/E/F). 

Clearly, other EEG components that are task-related can be isolated and could potentially be used to ``tag" BOLD data. The sliding window linear classification used here acts to reduce the EEG data along a dimension that categorizes stimulus evidence; however, this could be replaced by any other data reduction technique, such as temporally windowed ICA or PCA. Variability along these component directions could then be used in the encoding model to link with the simultaneously collected BOLD data. The choice of data reduction technique (i.e. feature space) would be highly dependent on the nature of the inferences.

In terms what our methodology is able to potentially reveal in terms of the spatiotemporal brain dynamics of perceptual decision making, we observe what appears to be reactivation of the pre-response network, spatiotemporal dynamics that would be masked using traditional fMRI analysis. Interestingly, the reactivation terminated in a network that included the MFG, SPL, and IPS, similar areas previously reported to be reactivated in metacognitive judgments of confidence in perceptual decisions \cite{Fleming2012,Steinhauser2010,Yeung2012}. In addition, these areas contributed the most to the correlation to our confidence proxy. Gherman and Philiastides \cite{Gherman2015} observed this network using a multivariate single-trial EEG approach, coupled with a distributed source reconstruction technique. Fleming et al.\cite{Fleming2012} and Heereman et al. \cite{Heereman2015} used BOLD fMRI to show that areas in this network negatively correlate with subjective certainty ratings.  Unique to our findings, we saw this reactivation on a single-trial basis after engagement of the ACC, which has been shown to be involved in decision monitoring \cite{Botvinick2001,Gherman2015}, and also observed the dynamic sequence leading up to this network reactivation. Our results showed that reactivation/replay occurred on a trial-to-trial basis after a decision, was stronger for difficult decisions, and correlated with a proxy for decision confidence (Fig. \ref{fig:Confidence}).

To ensure that our findings were not simply an artifact of a stimulus-locked analysis, we performed the same analysis response-locked.  We found that the reactivation, both its correlation with the confidence proxy and the cortical areas implicated, to be consistent with the stimulus locked analysis (see Fig. S6). In addition we found that the reactivation clearly begins pre-response and extends through and past the timing of the behavioral response (Fig. S7). Thus our complete analysis suggests that while in the process of making and executing a perceptual decision, the brain may replay activity to develop a metacognitive representation of its decision.

We note several caveats in terms of the interpretation of our novel results as they relate to perceptual decision-making.  One potential criticism is that the variability we discriminate and encode in the BOLD is related to the stimulus-evidence and not a decision variable such the choice (e.g. trial-to-trial variability of a subject’s decision of a face, car or house). Though our previous work has shown that there are EEG components that are sensitive to stimulus category or choice, these components tend to be isolated at specific time windows \cite{Philiastides2006,Philiastides2006d}. The level of stimulus evidence, on the other hand, spans most of the trial and thus can be used to provide a more complete picture of brain dynamics during the task. Ultimately the reactivation we observe in these dynamics is intriguing and thus we conducted additional analyses that points to a novel ``replay" hypothesis, namely that the reactivation is a replay of activity associated with the human brain generating, on a trial-to-trial basis, a meta-cognitive judgment of confidence immediately after the decision.  We note that this hypothesis is possible only by linking the EEG and BOLD data.  

Of course additional experiments must be done to more thoroughly test our ``replay" hypothesis.  For example future work will investigate other measures of confidence that are more direct than the proxy we estimated from the DDM.  We will also consider different types of rapid perceptual decision-making tasks to investigate whether the results we see generalize across stimulus types and/or the modality of stimulus presentation. In general, we have shown that simultaneously acquired EEG/fMRI data enables a novel non-invasive approach to visualize high resolution spatial and temporal processing in the human brain with the potential for providing a more comprehensive understanding of the neural basis of complex behaviors.
